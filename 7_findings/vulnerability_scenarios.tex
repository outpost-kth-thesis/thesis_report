% This section limits defines the scope of the thesis and reasons why vulnerability scenarios are necessary
\section{Scope of Vulnerability Scenarios}
When constructing an ensemble stack, it is beneficial to decompose the broader concept of ``vulnerability in a web server" into more manageable sub-tasks. Since the ensemble stack comprises AI models, each specializing in a distinct sub-task, this decomposition aligns with the fundamental design of the system. 

Although various applications exist for scanning web servers, they often fail to conduct exhaustive searches [citation-needed]. The scanner currently employed by the host company, for instance, lacks comprehensiveness. This limitation arises because black-box testing necessitates a ``human flair," making full automation particularly challenging. Additionally, the number of potential vulnerability scenarios is virtually limitless, rendering the development of a universal automated scanner impractical. 

And on top of that, applications have the potential to contain zero-day vulnerabilities that a group of malicious actors might be aware of. Automating the detection of such vulnerabilities using the current technology, including and excluding artificial intelligence, is quite difficult. Given these constraints, we refine the scope of our thesis to scenarios that satisfy the following criteria:

\begin{itemize}
    \item The scenario presents a meaningful scope for AI application. We define a meaningful scope as a scope where it is possible to use a trainable AI model that would otherwise be difficult to automate without the use of AI.
    \item The detection of the vulnerability is feasible with the technology available at the time of this writing.
    \item Given the vulnerability scenario, there must be an existing method by which a human can detect it.
\end{itemize}


\section{Scenarios covered}
\subsection{Vulnerable code detection}
A simple analogy is that of an experienced programmer ``vibe-checking'' potentially vulnerable code. The main methodology for tackling this sub-task is to incorporate pattern-matching in the target-code using a dataset consisting of common coding mistakes when writing JavaScript.



\subsection{Client-side vulnerabilities}
We describe client-side vulnerabilities as cases where the front-end code, that being the HTML and CSS, excluding JavaScript, might give away relevant information to a malicious attacker.

\section{Detecting vulnerabilities using Javascript}